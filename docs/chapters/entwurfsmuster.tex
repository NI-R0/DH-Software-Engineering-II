\section{\href{https://sourcemaking.com/design_patterns/chain_of_responsibility}{Entwurfsmuster}}
Das Entwurfsmuster \q{Chain-of-Responsibility} (kurz: CoR) dient dazu, die Bearbeitung von Anfragen der Reihe nach auf mehrere Objekte zu verteilen. Dabei wird eine Kette von Objekten gebildet, wobei jedes Objekt die Möglichkeit hat, die Anfrage zu bearbeiten und/oder sie an das nächste Objekt weiterzugeben. 


Der Einsatz von CoR ermöglicht es, Anfragen zu verarbeiten, ohne dass Sender und Empfänger der Nachricht fest verknüpft sind. Durch diese Art von Entkopplung können neue Handler-Objekte ohne weiteres zur Kette hinzugefügt, geändert oder ganz entfernt werden, ohne dabei das gesamte System anpassen zu müssen. Dies fördert in der Regel sowohl die Wartbarkeit, als auch die Übersichtlichkeit des Codes der gesamten Kette. Durch klar definierte Verantwortlichkeiten der einzelnen Handler-Objekte wird außerdem die Implementierung der Anfragebearbeitung vereinfacht und besser strukturiert. Bei komplexen Ketten mit vielen verschiedenen Handlern kann es allerding dazu kommen, dass die Wartbarkeit der wieder Kette abnimmt. Des weiteren besteht die Gefahr, dass Anfragen die gesamte Kette durchlaufen, bevor sie bearbeitet werden, was zu einer ineffizienten Verarbeitung führen kann.\\


\noindent{}In der Anwendung wird das Entwurfsmuster beispielsweise bei der Verarbeitung von REST-Anfragen verwendet: Vom REST-Controller wird eine Anfrage über den ApplicationService an das DomainRepository weitergeleitet, welches die Anfrage an das SpringDataRepository weitergibt. Hier wird die Anfrage abschließend bearbeitet, bevor die Response die gesamte Kette zurückgereicht wird, bis sie wieder beim REST-Controller ankommt. Insgesamt wurde das CoR-Muster drei mal in der Anwendung umgesetzt, einmal für jeden REST-Controller.